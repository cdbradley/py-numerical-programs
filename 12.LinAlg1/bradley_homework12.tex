\documentclass[11pt]{article}
\usepackage{amsmath}

\begin{document}
\title{Lesson 12}
\author{Colt Bradley}
\date{}
\maketitle

\section{Introduction}
The purpose of his lesson is to learn how to numerically solve systems of equations. The first step is to re-arrange systems of equations into matrix equations, like the example that follows:
\begin{subequations}
\begin{align}
2x +7z = 26 \\
3y +5z = 45 \\
x+3y-z = 0 \\
 \left( \begin{array}{ccc}
2 & 0 & 7 \\
0 & 3 & 5 \\
1 & 3 & -1 \end{array} \right)
 \left( \begin{array}{ccc}
x \\
y \\
z \end{array} \right) = 
 \left( \begin{array}{ccc}
26\ \\
45 \\
0\end{array} \right)
\end{align}
\end{subequations} 
In this way, any system of $n$ equations and variables can be represented as a $n \times n$ matrix equation. In python, there is a simple command that solves these systems. We'll demonstrate in both of the examples below. 

\section{Statics}
The first example is a simple statics problem with a mass connected to a wall by a string. There are three unknowns, the x component of the force $F_x$, the y-component of the force $F_y$, and tension in the string $T$. We can write a system of equations for these forces as follows:   
\begin{subequations}
\begin{align}
F_x - T \cos\theta = 0 \\
T_y + T \sin \theta - mg = 0 \\
(T \sin \theta)(l / 2) - F_y(l/2)= 0
\end{align}
\end{subequations}
Next, we move constants to the left and identify which terms go with which variables. This will become our matrix equations. 

\begin{equation}
 \left( \begin{array}{ccc}
1 & 0 & - \cos \theta \\
0 & 1 & \sin \theta \\
0 & - \frac{l}{2} & \sin \theta \frac{l}{2} \end{array} \right)
 \left( \begin{array}{ccc}
F_x \\
F_y \\
T \end{array} \right) = 
 \left( \begin{array}{ccc}
0 \\
mg \\
0\end{array} \right)
\end{equation}

Writing a code in python, we solve using the given constants. We let $m = 14$ kg, $l = 1.2$ m, $g = 9.8$ m/s, and $\theta = 35^{\circ}$. The answer we get for these values follows. 

\begin{subequations}
\begin{align}
F_x = 144.8 \\
F_y = 68.6 \\
T = -1.3 
\end{align}
\end{subequations} 

\section{Kirchoff's Laws}
In this example, our system is solving a circuit using Kirchoff's laws. This yeilds four equations; one for each loop and one for conservation of current. 
\begin{subequations}
\begin{align}
\mathcal{E}_1 - I_1 R_1 - I_3 R_3 = 0 \\
\mathcal{E}_2 - I_2 R_2 - I_3 R_3 = 0\\
\mathcal{E}_2 - I_2 R_2 +I_1 R_1 - \mathcal{E}_1 = 0 \\
I_1 +I_2-I_3= 0
\end{align}
\end{subequations}
We can quickly see that the conservation of current equation is essential, since looking at \ref{singular} we see that the matrix is singular. To avoid this, choose any two of the first three equations plus the current conservation equation. 

\begin{equation}
\left( \begin{array}{ccc}
0 & R_2 & R_3 \\
R_1 & 0 & R_3 \\
R_1 & -R_2 & 0 \end{array} \right)
\label{singular}
\end{equation}
We`ll then solve these like we did in the previous example, moving anything that doesn't rely on our unknown currents (any $\mathcal{E}$) to the right hand side then building a matrix equation out of it. 
\begin{equation}
 \left( \begin{array}{ccc}
R_1 & 0 & R_3 \\
0 & R_2 & R_3 \\
1 & 1 & -1 \end{array} \right)
 \left( \begin{array}{ccc}
I_1 \\
I_2 \\
I_3 \end{array} \right) = 
 \left( \begin{array}{ccc}
\mathcal{E}_1\ \\
\mathcal{E}_2 \\
0\end{array} \right)
\end{equation}
This equation can be solved with almost identical code to the first, and when we solve we get. 
\begin{subequations}
\begin{align}
I_1 = .062 \\
I_2 = .025 \\
I_3 = .088
\end{align}
\end{subequations}
\section{Code}
\begin{verbatim}
#Colt Bradley
#2.25.16
#Homework 12

#import modules
import numpy as n

#define variables
m = 14
l = 1.2
g = 9.8
theta = 35

#Define the matricies
big = n.matrix([[1, 0 ,-n.cos(theta)],[0,1,n.sin(theta)],\
[0, -l/2., n.sin(theta)*l/2]])
col = n.matrix([[0],[m*g],[0]])

#use linear algebra package, print result
print n.linalg.solve(big,col)

############################################################################
#Part 2
############################################################################

#define variables, emf in Volts and r in ohms
emf_1 = 12
emf_2 = 9
r_1 = 100
r_2 = 130
r_3 = 65

#define matricies
res = n.matrix([[r_1,0,r_3],[0,r_2,r_3],[1,1,-1]])
emfs = n.matrix([[emf_1],[emf_2],[0]])

#use linear algebra package, print result
print n.linalg.solve(res,emfs)
\end{verbatim}

\end{document}