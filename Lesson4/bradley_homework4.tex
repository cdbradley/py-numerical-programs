\documentclass[11pt]{article}
\usepackage{graphicx}

\begin{document}
\title{Homework 4}
\author{Colt Bradley}
\date{}
\maketitle

\section{Introduction}
Homework 4 required us to run a sequence of commands to play with selecting certain elements from lists. A copy of the code is included in the end, but where relevant lines of code are inserted throughout.

\subsection{Problem 1}
We assigned a list using \emph{n.arange} and printed it. 
\begin{verbatim}
[-2  0  2  4  6  8 10]
\end{verbatim}

\subsection{Problem 2}
We looked at specific elements or subsets of a list. Notice that you can use the ``:" symbol to select all elements before a certain number. 
\begin{verbatim}
a = [19,4,-7,13,55,-12,16,8]
a[3] = 13
a[3:] = [13, 55, -12, 16, 8]
a[:3] = [19, 4, -7]
\end{verbatim}

\subsection{Problem 3}
The third problem was looking at what type of variable you ended up with after preforming certain operations. The default would be a float. For example, $5^3$ is an integer since both are defined as integers, but the answer of both $5.^3$ and $5^3.$ are floats, even though the answer itself doesn't change. 

\subsection{Problem 4}
The idea was very similar to problem 3. An integer and a float added or multiplied together is a float. 
\begin{verbatim}
7 is an integer
4.2 is a float
7+4.2 is a float
7*4.2 is a float
\end{verbatim}

\subsection{Problem 5}
Again, this problem was similar to 3 and 4, instead dealing with complex. 
\begin{verbatim}
3.4 is a float
2+8j is complex
3.4 + (2+8j) is complex
3.4*(2+8j) is complex
\end{verbatim}

\subsection{Problem 6}
The {:.2f} in the command below tells the program to display only two decimals.
\begin{verbatim}
"x = {:.2f}".format(x) yields x=3.33
\end{verbatim}

\subsection{Problem 7}
\subsubsection{Part 1}
We are required to ask the user's name and return their initials. Prompting for their name is simple, we use .split() command to same a first and last name. Then we take the first character of each and combine them, and use a print command to display.

Alternatively, we could have used .format like in problem 6 to take the first character. 

\begin{verbatim}
#prompt user for first and last name
name1,name2 = raw_input("What\'s your first and last name? ").split()

#grab the initials and string combine them
initials = name1[0]+name2[0]

#print necessary info
print "Hello {} {}. Your initials are {}" .format(name1,name2,initials)
\end{verbatim}

\subsubsection{Part 2}
Now, we were required to prompt for an initial velocity and position of a projectile and return the time it would take for the projectile to hit the ground. 
\begin{equation}
t_1 = \frac{v_0}{g} \label{time up}
\end{equation}
\begin{equation}
y_1 = v_0 t - \frac{1}{2}gt^2 \label{y up}
\end{equation}
\begin{equation}
t_2 = \sqrt{\frac{2(y_1 + y_0)}{g}} \label{time}
\end{equation}
To do this, we first solve for the time it takes for the projectile to peak. We'll use \ref{time up} to do this, then plug the value into \ref{y up} to determine how far above the initial point the particle went. We then calculate the time it takes for the particle to fall using \ref{time}. The value $t_1 + t_2$ will give the total time down. The code for this is below.  

\section{Code}
\begin{verbatim}
#Colt Bradley
#1.19.16
#Lesson 4 Homework

#import modules 
import numpy as n

#Question 1
x = n.arange(-2,12,2)
print x

#Question 2
a = [19,4,-7,13,55,-12,16,8]
print a[3]
print a[3:]
print a[:3]

#question 3
x = 5**3
print type(x)
x = 5.**3
print type(x)
x = 5**3.
print type(x)

#question 4
x = 7
y = 4.2
print type(x)
print type(y)
print type(x+y)
print type(x*y)

#Question 5
x = 3.4
y = 2+8j
print type(x)
print type(y)
print type(x+y)
print type(x*y)

#Question 6
x=10./3.
print "x = {:.2f}".format(x)

#Question 7
#prompt user for first and last name
name1,name2 = raw_input("What\'s your first and last name? ").split()

#grab the initials and string combine them
initials = name1[0]+name2[0]

#print necessary info
print "Hello {} {}. Your initials are {}" .format(name1,name2,initials)

#now ask for an intial hgiht and velocity of a 1-d particle in gravity
y0 = raw_input("What's the height of a projectile? ")
y0 = float(y0)
v0 = raw_input("What's the initial velocity of that projectile?")
v0 = float(v0)

#essential values
g = 9.8


#preform calculation
t1 = v0/g
y1 = -.5*g*t1**2 + v0*t1
t2 = n.sqrt(2*(y1+y0)/g)
time = t1 + t2

#print total flight time
print "The total time of flight for the object is t = {:.3f} s." .format(time)
\end{verbatim}

\end{document}