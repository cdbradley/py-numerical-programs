\documentclass[11pt]{article}
\usepackage{graphicx}

\begin{document}
\title{Homework 6}
\author{Colt Bradley}
\date{}

\maketitle

\section{Part 1: Explanation}
The point of this part was to write a set of functions in another file and import them into your python code. They were simple functions, just a factorial and power function. The trick; however, was in how they returned values. To make this right, it was important to return the right value in the functions file so that the returned value in the code was the correct type (a float as opposed to a `none' type). 

The overall code imports my functions like any package, ask the user for two numbers, and combines them using the defined functions. 

\section{Part 2: Explanation}
This exercise dealt with writing data to a file. After a user gives a number `n', the programs creates an $n \times n$ multiplication table. I made this by creating an $(n+1)\times (n+1)$ table of zeroes, then writing the values over them. This method allowed me to put the number $n$ in the top left corner and also yielded a nice border of zeroes. The actual multiplication was done using two for loops iterating through a list of all integers between 1 and $n$. 

\section{Part 3: Explanation}
The goal of the final exercise was to work with other ways of interacting with text files, especially using strings. To complete it, I created a list of fruits and saved it as a text file. I then prompted the user for a fruit, which is written to a new text file along with the list from the other text file. This list is then read and printed on the screen after the phrase, ``My favorite fruits are...''. The coding was fairly straightforward for this exercise, but the commands and the difference between `read' and `write' took a bit to get used to.  

\section{Homework Code}

\begin{verbatim}
#Colt Bradley
#1.31.16
#Lesson 6 Homework

#import essential packages
import numpy as n
import bradley_functions as f

###############
#HW 1
#asks for a base, exponent, combines in user-defined functions
###############

#ask the user for input for each of these values
N = raw_input("Give me an integer as a base: ")
N = int(N)
r = raw_input("Give me an integer as an exponent: ")
r = int(r)

#Use functions and user arguments to answer
print f.power(f.factorial(N),r)

###############
#HW 2
#creates nxn dimension multiplication table and saves it
##############

#Interact with the user
print "We\'ll create a square multiplication table."
size = raw_input("What size should it be? ")
size = int(size)

#Create matrix placeholder
mat = n.zeros((size+1,size+1))
#Create list up to size
rng = range(0,size+1)

#Create "Titles" for the matrix
mat[0,0] = size

#fills in the interior of the matrix
#exterior will be lined by zeros, size will be in top right corner
for i in range(1,size+1):
    for j in range(1,size+1):
        mat[i,j] = rng[j]*rng[i]

n.savetxt("mult_table.txt",mat)

################
#HW 3
#ask for a fruit, write it to a list
###############

#open fruit file
fruits = open("new_fruits.txt","w")
fruits1 = open("fruits.txt","r")

#ask user for a fruit
user_fruit = raw_input("Enter a fruit: ")

#write the old and new fruits to the new file
string = fruits1.read()
fruits.write(string)
fruits.write(user_fruit)
fruits.close()
fruits1.close()

#open new fruits file and save a string
fruitss = open("new_fruits.txt","r")
string = fruitss.read()
    
#print necessary information
print "My favorite fruits are..."
print string
\end{verbatim}

\section{Functions Code}

\begin{verbatim}
#Colt Bradley
#1.28.16
#some functions from lesson 6

def factorial(n):
    x=1
    for i in range(2,n+1):
        x = x*i
    return x
   
    
def power(base,exponent=2):
    return base**exponent
\end{verbatim}




\end{document}