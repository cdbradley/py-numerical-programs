\documentclass[11pt]{article}

\begin{document}
\title{Homework 24}
\author{Colt Bradley}
\date{}
\maketitle

\section{Description}

This code involves several functions. The first computes the pentagonal numbers given an integer number. The formula is $n(3n-1)/2$. We compute the pentagonal numbers for the set $[1,50]$ and add each to a list. The function ``compare'' then compares the sum of two pentagonal numbers with the other pentagonal numbers. If true, the two that sum are added to another list. We get some duplicates using this method, so we filter using sets. 

\section{Code}
\begin{verbatim}
#Colt Bradley
#3.22.2016
#Lesson 24


#############################################################################
#functions and modules
#############################################################################

#import modules
import numpy as n
import pylab as p

def pentagonal(n):
    return n*(3*n-1)/2.
    
#Checks to see if the sum of two elements of list a and b is in list c, then
#prints that list
def compare(a,b,c):
    pents = []
    K = []
    N = []
    for i in a:
        for k in b:
            for n in c:
                summ = k+n
                
                if summ == i:
                    pents.append(summ)
                    K.append(k)
                    N.append(n)
                    break
    pairs = zip(K,N)
    return pairs  

#############################################################################
#Exercise 1
#############################################################################

#first, create a list of all pentagonal numbers
n1 = n.linspace(1,50)
pent = []
for i in n1:
    pent.append(pentagonal(i))
    
#creates a longer list of pentagonal numbers for comparison beyond n=50
n2 = n.linspace(1,100,100)
pent2 = []
for i in n2:
    pent2.append(pentagonal(i))
    
#now, compare using compare function
A = compare(pent2,pent,pent)

#Delete the commutative duplicates using sets
seen = set()
new_pents = []
for item in A:
    item_set = frozenset(item)
    if item_set not in seen:
        new_pents.append(item)
        seen.add(item_set)
print new_pents
\end{verbatim}


\end{document}