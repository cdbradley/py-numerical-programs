\documentclass[11pt]{article}

\begin{document}
\title{Homework 5}
\author{Colt Bradley}
\date{}
\maketitle

\section{Introduction}
The assignment was to create a program that prompted for a number, then determined all the prime numbers less that the given number and printed a statement. 

\section{Logic}
My code prompts the user for an integer, then stores it as an integer. We then handle the case of 1, 2, and 3 separately using if statements to make it a bit easier. The bulk of the code uses two nested while loops, the first running through every integer between 4 (since we already checked 1-3) and the inputted integer. The second while loop checks if that number is divisible by any number smaller than it. If it is, a marker is added to a testing list. At the end of the loop, if that list is empty the number is added to a list of primes (which already contains 2 and 3). If the list isn't empty, it's cleared and we move on to the next number.

At the end, we have a list of primes called ``primes." To see if our integer was prime, we check if it equals the last value of this list. If it does, we print ``Congratulations! The number you chose was prime!" If it doesn't, we print "Sorry, the number you chose was not prime." Either way, we print our list of primes. 

\section{Python Code}
\begin{verbatim}
#Colt Bradley
#1.26.16
#Lesson 5 Homework

#import essential modules 
import numpy as n

#ask user for an integer
integer = raw_input("Give me any integer ")
integer = int(integer)

#sets up the list to hold prime numbers
primes = [2,3]

#first, we take care of special cases. 1, 2, and 3 are cases that if we take care 
#of now it will be easier later            
if integer == 1:
    print "Sorry, the number you chose was not prime."
elif integer == 2:
    print "Congratulations! The number you chose was prime!"
elif integer == 3:
    print primes
    print "Congratulations! The number you chose was prime!"

#here's the bluk of the code. Basically, the first while loop runs through every  
#number up to and including whatever inputted number. The second while loop checks 
#what numbers divide that number and append it to a list. Remember that empty  
#lists are false, so if the list is false then we append our test number to our  
#prime list and iterate. If true, we just interate. at the end, we check if the 
#final value of the list is equal to the inputted value, and if so we declare 
#that it is prime and print our list of primes. 
else:
    r = 4
    while r <= integer+1:
        m = 2
        test = []
        while m < r:
            if (r%m) == 0:
                test.append(m)
                m = m+1
            else:
                m = m+1
        if not test:
            primes.append(r)
            r = r+1
        else: r = r+1
    print primes
    if primes[-1] == integer:
        print "Congratulations! The number you chose was prime!"
    else:
        print "Sorry, the number you chose was not prime."
\end{verbatim}


\end{document}